\documentclass{jarticle}

\usepackage[dvipdfmx]{graphicx}
\usepackage{float}
\usepackage{url}

\title{放射線の計測}
\author{2511198 肥田幸久 \\ 共同実験者 \\ }
\date{2025年10月10日作成}

\begin{document}
\maketitle


\section{実験の目的}

GM計数管を用いて, 放射性原子の放射性崩壊の法則と物質による放射線の吸収について調べる.


\section{実験の原理}

\subsection{放射性原子の崩壊}

原子核は陽子と中性子から構成されているが, 原子核の質量は, 陽子と中性子の質量の和よりも小さく, この質量の差を質量欠損という.
これは, 原子核を構成する陽子と中性子が結合しているためであり, 質量欠損$\Delta m$ に相当するエネルギー$\Delta mc^2$を結合エネルギーという.

ある原子核が安定でない場合, その原子核は放射性崩壊を起こし, より安定な原子核に変化する.
このとき, $\alpha$線, $\beta$線, $\gamma$線などの放射線が放出される.
この現象が放射性原子の崩壊である.

\subsection{放射性崩壊の計数値の分布}

放射性原子の崩壊は, ある一定の確率で起こる現象であり, 個々の原子については, いつ崩壊するかを予測することはできない.
放射性原子が崩壊する確率は, その原子核の種類に依存する.

ある核種$\mathrm{A}$が時刻$t$において$\mathcal{N}$個あるとする.
微小時間$\mathrm{d}t$の間に崩壊を起こして他の核種$\mathrm{B}$に変化する数を$\mathrm{d}N$とすると, 単位時間あたりの崩壊数$\mathrm{d}\mathcal{N}/\mathrm{d}t$は$\mathcal{N}$に比例し, 比例定数(崩壊定数)を$\lambda$とすると, 次のように表される.
\begin{equation}
  \frac{\mathrm{d}\mathcal{N}}{\mathrm{d}t} = -\lambda \mathcal{N}
\end{equation}

時刻$t=0$における核種$\mathrm{A}$の数を$\mathcal{N}_0$とすると, 時刻$t$における核種$\mathrm{A}$の数は平均的に次のようになる.
\begin{equation}
  \mathcal{N}(t) = \mathcal{N}_0 e^{-\lambda t}
\end{equation}

この数が半分になる時間を半減期$\tau$といい, 次の関係が成り立つ.
\begin{equation}
  \tau = (\log_e 2)/\lambda
  = 0.693/\lambda
\end{equation}

次に, $\lambda t$が$1$に比べて十分小さいとき, $e^{-\lambda t}$はテイラー展開により,
\begin{equation}
  e^{-\lambda t} \cong 1 - \lambda t
\end{equation}
と近似できる.
したがって, 時刻$t$における核種$\mathrm{A}$の数は次のように表される.
\begin{equation}
  \mathcal{N}(t)=\mathcal{N}_0-\mathcal{N}_0\lambda t
\end{equation}
この式から, 放射性崩壊の数を時間$t$にわたって観測して得られる計数値$N$の平均値$\overline{N}$は
\begin{equation}
  \overline{N} = \mathcal{N}_0\lambda t
\end{equation}
と期待できる.

そしてその計数値$N$の確率$P(N)$は, 次のポアソン分布に従う.
\begin{equation}
  P(N) = \frac{\overline{N}^N}{N!}e^{-\overline{N}}, \quad
  \sum_{N=0}^{\infty} P(N) = 1
\end{equation}
また, 平均値$\overline{N}$が大きくなると, ポアソン分布は次のガウス分布(正規分布)に近づく.
\begin{equation}
  G(N) = \frac{1}{\sqrt{2\pi\overline{N}}} \exp \left\{-\frac{(N-\overline{N})^2}{2\overline{N}}\right\}
\end{equation}

\subsection{$\beta$線の吸収}



\section{実験方法}


\section{実験結果}


\section{考察}



\begin{thebibliography}{99}

  \bibitem{ref1} 参考文献1の情報
  \bibitem{ref2} 参考文献2の情報
  \bibitem{ref3} 参考文献3の情報

\end{thebibliography}

\end{document}